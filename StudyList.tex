% Options for packages loaded elsewhere
\PassOptionsToPackage{unicode}{hyperref}
\PassOptionsToPackage{hyphens}{url}
\PassOptionsToPackage{dvipsnames,svgnames,x11names}{xcolor}
%
\documentclass[
  letterpaper,
  DIV=11,
  numbers=noendperiod]{scrartcl}

\usepackage{amsmath,amssymb}
\usepackage{lmodern}
\usepackage{iftex}
\ifPDFTeX
  \usepackage[T1]{fontenc}
  \usepackage[utf8]{inputenc}
  \usepackage{textcomp} % provide euro and other symbols
\else % if luatex or xetex
  \usepackage{unicode-math}
  \defaultfontfeatures{Scale=MatchLowercase}
  \defaultfontfeatures[\rmfamily]{Ligatures=TeX,Scale=1}
\fi
% Use upquote if available, for straight quotes in verbatim environments
\IfFileExists{upquote.sty}{\usepackage{upquote}}{}
\IfFileExists{microtype.sty}{% use microtype if available
  \usepackage[]{microtype}
  \UseMicrotypeSet[protrusion]{basicmath} % disable protrusion for tt fonts
}{}
\makeatletter
\@ifundefined{KOMAClassName}{% if non-KOMA class
  \IfFileExists{parskip.sty}{%
    \usepackage{parskip}
  }{% else
    \setlength{\parindent}{0pt}
    \setlength{\parskip}{6pt plus 2pt minus 1pt}}
}{% if KOMA class
  \KOMAoptions{parskip=half}}
\makeatother
\usepackage{xcolor}
\setlength{\emergencystretch}{3em} % prevent overfull lines
\setcounter{secnumdepth}{-\maxdimen} % remove section numbering
% Make \paragraph and \subparagraph free-standing
\ifx\paragraph\undefined\else
  \let\oldparagraph\paragraph
  \renewcommand{\paragraph}[1]{\oldparagraph{#1}\mbox{}}
\fi
\ifx\subparagraph\undefined\else
  \let\oldsubparagraph\subparagraph
  \renewcommand{\subparagraph}[1]{\oldsubparagraph{#1}\mbox{}}
\fi


\providecommand{\tightlist}{%
  \setlength{\itemsep}{0pt}\setlength{\parskip}{0pt}}\usepackage{longtable,booktabs,array}
\usepackage{calc} % for calculating minipage widths
% Correct order of tables after \paragraph or \subparagraph
\usepackage{etoolbox}
\makeatletter
\patchcmd\longtable{\par}{\if@noskipsec\mbox{}\fi\par}{}{}
\makeatother
% Allow footnotes in longtable head/foot
\IfFileExists{footnotehyper.sty}{\usepackage{footnotehyper}}{\usepackage{footnote}}
\makesavenoteenv{longtable}
\usepackage{graphicx}
\makeatletter
\def\maxwidth{\ifdim\Gin@nat@width>\linewidth\linewidth\else\Gin@nat@width\fi}
\def\maxheight{\ifdim\Gin@nat@height>\textheight\textheight\else\Gin@nat@height\fi}
\makeatother
% Scale images if necessary, so that they will not overflow the page
% margins by default, and it is still possible to overwrite the defaults
% using explicit options in \includegraphics[width, height, ...]{}
\setkeys{Gin}{width=\maxwidth,height=\maxheight,keepaspectratio}
% Set default figure placement to htbp
\makeatletter
\def\fps@figure{htbp}
\makeatother

\KOMAoption{captions}{tableheading}
\makeatletter
\makeatother
\makeatletter
\makeatother
\makeatletter
\@ifpackageloaded{caption}{}{\usepackage{caption}}
\AtBeginDocument{%
\ifdefined\contentsname
  \renewcommand*\contentsname{Table of contents}
\else
  \newcommand\contentsname{Table of contents}
\fi
\ifdefined\listfigurename
  \renewcommand*\listfigurename{List of Figures}
\else
  \newcommand\listfigurename{List of Figures}
\fi
\ifdefined\listtablename
  \renewcommand*\listtablename{List of Tables}
\else
  \newcommand\listtablename{List of Tables}
\fi
\ifdefined\figurename
  \renewcommand*\figurename{Figure}
\else
  \newcommand\figurename{Figure}
\fi
\ifdefined\tablename
  \renewcommand*\tablename{Table}
\else
  \newcommand\tablename{Table}
\fi
}
\@ifpackageloaded{float}{}{\usepackage{float}}
\floatstyle{ruled}
\@ifundefined{c@chapter}{\newfloat{codelisting}{h}{lop}}{\newfloat{codelisting}{h}{lop}[chapter]}
\floatname{codelisting}{Listing}
\newcommand*\listoflistings{\listof{codelisting}{List of Listings}}
\makeatother
\makeatletter
\@ifpackageloaded{caption}{}{\usepackage{caption}}
\@ifpackageloaded{subcaption}{}{\usepackage{subcaption}}
\makeatother
\makeatletter
\@ifpackageloaded{tcolorbox}{}{\usepackage[many]{tcolorbox}}
\makeatother
\makeatletter
\@ifundefined{shadecolor}{\definecolor{shadecolor}{rgb}{.97, .97, .97}}
\makeatother
\makeatletter
\makeatother
\ifLuaTeX
  \usepackage{selnolig}  % disable illegal ligatures
\fi
\IfFileExists{bookmark.sty}{\usepackage{bookmark}}{\usepackage{hyperref}}
\IfFileExists{xurl.sty}{\usepackage{xurl}}{} % add URL line breaks if available
\urlstyle{same} % disable monospaced font for URLs
\hypersetup{
  pdftitle={Draft Studylist Final},
  colorlinks=true,
  linkcolor={blue},
  filecolor={Maroon},
  citecolor={Blue},
  urlcolor={Blue},
  pdfcreator={LaTeX via pandoc}}

\title{Draft Studylist Final}
\author{}
\date{}

\begin{document}
\maketitle
\ifdefined\Shaded\renewenvironment{Shaded}{\begin{tcolorbox}[sharp corners, borderline west={3pt}{0pt}{shadecolor}, frame hidden, enhanced, interior hidden, boxrule=0pt, breakable]}{\end{tcolorbox}}\fi

\hypertarget{k-nearest-neighbors}{%
\subsection{k-Nearest Neighbors}\label{k-nearest-neighbors}}

\begin{itemize}
\item
  You must manually calculate the Euclidian Distance between a testing
  record and 3 observations from the training dataset for two predictor
  variables and then find the nearest neighbor for k=1.
\item
  Euclidean Distance between a testing record and 9 observations is
  given together with the outcome for the 9 observations. You have to
  manually find the 3 nearest neighbors and make a prediction.
\item
  In a diagram with two predictors for a binary classification that
  shows both classes color-coded for the training data and also shows
  one record to be classified. Use k=5 and k=1 to classify the record.
\end{itemize}

\hypertarget{linear-regression}{%
\subsection{Linear Regression}\label{linear-regression}}

\begin{itemize}
\item
  Calculating regression coefficients: not relevant
\item
  Understand (!) the logic of the MSE (e.g., identify the error for an
  individual prediction in the formula, identify an individual
  prediction in the formulas, ee why MSE is the mean of squared errors)
\item
  Understand the role of the betas (parameters and why their choice
  determines the MSE)
\item
  interpret the coeficients and the p-values from a fitted regression
\end{itemize}

\hypertarget{logistic-regression}{%
\subsection{Logistic Regression}\label{logistic-regression}}

\begin{itemize}
\item
  When a graph of a linear or logistic regression line is given in a
  diagram, find the probabilities for YES/NO
\item
  When a graph of a linear or logistic regression line is given in a
  diagram, find the decision boundary (e.g.~every income greater 100k is
  predicted as YES, smaller 100k results in NO).

  \begin{itemize}
  \tightlist
  \item
    which data points are predicted Yes which NO
  \item
    which data points are predicted correctly which are predicted
    incorrectly
  \end{itemize}
\end{itemize}

\hypertarget{neural-networks}{%
\subsection{Neural Networks}\label{neural-networks}}

\begin{itemize}
\tightlist
\item
  Make a prediction when a Neural Network with parameters (\(\beta s\))
  is given

  \begin{itemize}
  \tightlist
  \item
    in algebraic form
  \item
    in a diagram
  \end{itemize}
\item
  Calculate the MSE for a prediction (testing observation),when a Neural
  Network with parameters (\(\beta s\)) is given

  \begin{itemize}
  \tightlist
  \item
    in algebraic form
  \item
    in a diagram
  \end{itemize}
\end{itemize}

\hypertarget{decision-tree}{%
\subsection{Decision Tree}\label{decision-tree}}

\begin{itemize}
\item
  Predict with a decision tree when spliting rules are given.

  \begin{itemize}
  \tightlist
  \item
    Classification model
  \item
    Regression model
  \end{itemize}
\item
  Understand the meaning of the meaning of the three numbers in each
  node.
\item
  Interpret a decision tree (terminal nodes and other nodes).
\item
  Understand a confusion matrix and be able to calculate accuracy,
  sensitivity, and specificity)
\item
  Know what a splitting variable and a splitting value are
\item
  Not on final: Gini impurity and variance decrease
\end{itemize}

\hypertarget{random-forest}{%
\subsection{Random Forest}\label{random-forest}}

\begin{itemize}
\item
  How does a Random Forest generates a prediction from multiple decision
  trees (you might have to calculate the Random Forest Prediction from 5
  or 10 trees.

  \begin{itemize}
  \tightlist
  \item
    Classification
  \item
    Regression
  \end{itemize}
\item
  How do we ensure the decision trees of a random forest different?
\item
  Random Subspace method
\item
  Bagging (i.e., Bootstrapping)
\item
  How can you create a bootstrap dataset from an original training
  dataset?
\end{itemize}

\hypertarget{interpretation}{%
\subsection{Interpretation}\label{interpretation}}

\begin{itemize}
\item
  Know the difference between

  \begin{itemize}
  \tightlist
  \item
    local/global interpretation
  \item
    model specific/ model agnostic
  \end{itemize}
\item
  Interpret a Variable importance Plot
\item
  SHAP values:

  \begin{itemize}
  \tightlist
  \item
    Understand why SHAP values are local/model agnostic interpreters
  \item
    Understand the relation between (when numerical values are give):

    \begin{itemize}
    \tightlist
    \item
      average prediction
    \item
      SHAZP values for all predictor variables for the observaion
    \item
      predicted value for that observation
    \end{itemize}
  \end{itemize}
\end{itemize}



\end{document}
